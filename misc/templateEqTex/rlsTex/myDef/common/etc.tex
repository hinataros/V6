\newcommand{\eqRef}[1]{
  (\ref{eq:#1})
}
\renewcommand{\figurename}{Fig.\ }
\newcommand{\figRefNum}[1]{
  \ref{fig:#1}
}
\newcommand{\figRef}[1]{
  Fig.~\ref{fig:#1}
}

%% \newcommand{\subFigRefChar}[1]{
%%   \!\!\!(\subref{fig:#1})
%% }
%% \newcommand{\subFigRef}[2]{
%%   Fig.~\ref{fig:#1}(\subref{fig:#2})
%% }

\renewcommand{\tablename}{Table }
\newcommand{\tabRef}[1]{
  Table~\ref{#1}
}
\newcommand{\tbRef}[1]{
  {Table\ \ref{tab:#1}}
}
\newcommand{\figsRef}[2]{
  {Fig.\ \ref{fig:#1}}$\sim$\ref{fig:#2}
}
\newcommand{\secRef}[1]{
  {\ref{sec:#1}}章 % amiyata
  % {\ref{sec:#1}}
}
\newcommand{\appRef}[1]{
  {付録\ref{app:#1}}章 % amiyata
  % {付録\ref{app:#1}}
}
\newcommand{\ssecRef}[1]{
  {\ref{ssec:#1}}節 % amiyata 
  % {\ref{ssec:#1}}
}

\newcommand{\applb}[1]{
  \label{app:#1}
}
\newcommand{\pr}{
  \protect
}

\newcommand{\nt}{
  \notag
}

\newcommand{\hl}{
  \hline
}
\newcommand{\hhl}{
  \hline\hline
}

\newcommand{\adj}{
  \mrm{adj}
}
\newcommand{\diag}{
  \mrm{diag}
}

\NewDocumentCommand\iR{m o}
{
  \IfNoValueTF{#2}
  {
    \in \Re^{{#1}}
  }
  {
    \in \Re^{#1\times#2}
  }
}

\newcommand{\bal}[2][]{
  \ifthenelse{\equal{}{#1}}{
    \begin{align}
      #2
    \end{align}
  }
  {
    \begin{align}\label{eq:#1}
      #2
    \end{align}
  }
}


% \NewDocumentCommand\bfig{m o m m}
% {
%   \begin{figure}[#1]
%     #3
%     \caption{#4}
%     \IfValueT{#2}{\label{fig:#2}}
%   \end{figure}
% }
% amiyata modify fig自体をminipageしたい時にキャプションが邪魔
\NewDocumentCommand\bfig{m o m o}
{
  \begin{figure}[#1]
    #3
    \IfValueT{#4}{\caption{#4}}
    \IfValueT{#2}{\label{fig:#2}}
  \end{figure}
}
\NewDocumentCommand\bfigf{m o m o}
{
  \begin{figure*}[#1]
    #3
    \IfValueT{#4}{\caption{#4}}
    \IfValueT{#2}{\label{fig:#2}}
  \end{figure*}
}

%% \NewDocumentCommand\fig{m o m m}
%% {
%%   \setcounter{figCount}{\value{figure}}
%%   \stepcounter{figCount}

%%   \begin{figure}[#1]
%%     #3
%%     \caption{#4}
%%     \IfValueT{#2}{\label{fig:#2}}
%%   \end{figure}
%% }

% \NewDocumentCommand\bmpsc{m m o m}
% {
%   \begin{minipage}[#1]{#2\linewidth}
%     #4
%     \vspace{-6mm}
%     \subcaption{}
%     \IfValueT{#3}{\label{fig:#3}}
%   \end{minipage}
% }
%amiaya modify 上記と同状況でキャプションを出したい
\NewDocumentCommand\bmpsc{m m o m o}
{
  \begin{minipage}[#1]{#2\linewidth}
    #4
    % \vspace{-6mm}
    \IfValueTF{#5}{\caption{#5}}{\subcaption{}}
    \IfValueT{#3}{\label{fig:#3}}
  \end{minipage}
}

\NewDocumentCommand\bmpnsc{m m o m}
{
  \begin{minipage}[#1]{#2\linewidth}
    #4
    \IfValueT{#3}{\label{fig:#3}}
  \end{minipage}
}

\NewDocumentCommand\ig{o m m}
{
  \centering
  \IfValueTF{#1}{
    \ifthenelse{\equal{w}{#1}}
    {
      \includegraphics[width=#2]{%
        #3%
      }
    }
    {
      \ifthenelse{\equal{h}{#1}}
      {
        \includegraphics[height=#2]{%
          #3%
        }
      }
      {
        \ifthenelse{\equal{s}{#1}}
        {
          \includegraphics[scale=#2]{%
            #3%
          }
        }
        {
          size option error!
        }
      }
    }
  } %T
  {
    \includegraphics[width=#2\linewidth]{%
      #3%
    }
  }
}

% \NewDocumentCommand\ig{m o}
% {
%   \centering
%   \includegraphics[width=#1\linewidth]{%
%     \figDir//chapter\arabic{chapter}/fig\arabic{chapter}.\arabic{figCount}#2.eps%
%   }
% }

%% \NewDocumentCommand\ig{m m}
%% {
%%   \centering
%%   \includegraphics[width=#1\linewidth]{%
%%     \figDir//chapter\arabic{chapter}/section\arabic{section}/subsection\arabic{subsection}/#2%
%%   }
%% }

\newcommand{\bmat}[1]{ % amiyata
  \begin{matrix}
    #1
  \end{matrix}
}

\newcommand{\bbmat}[1]{
  \begin{bmatrix}
    #1
  \end{bmatrix}
}

\NewDocumentCommand\btb{m o m o}
{
  \begin{table}[#1]
    \IfValueT{#4}{\caption{#4}}
    \centering
    #3
    \IfValueT{#2}{\label{tab:#2}}
  \end{table}
}

\NewDocumentCommand\btbl{m m}
{
  \begin{tabular}{#1}
    #2
  \end{tabular}
}

\newcommand{\bcss}[1]{
  \begin{cases}
    #1
  \end{cases}
}

\newcommand{\bit}[1]{
  \begin{itemize}
    #1
  \end{itemize}
}

\newcommand{\benum}[1]{
  \begin{enumerate}
    #1
  \end{enumerate}
}

\newcommand{\ssm}[1]{
  [#1^\times]
}

\newcommand{\ola}[1]{
  \protect\overleftarrow{#1}
}
\newcommand{\ora}[1]{
  \protect\overleftarrow{#1}
}

\newcommand{\sectionl}[2]{
  \section{#1} \label{sec:#2}
}
\newcommand{\ssectionl}[2]{
  \subsection{#1} \label{ssec:#2}
}
\newcommand{\sssectionl}[2]{
  \subsubsection{#1} \label{sssec:#2}
}
